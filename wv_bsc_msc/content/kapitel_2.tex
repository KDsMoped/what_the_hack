% Kapitel
\chapter{Stile}

\epigraph{\glqq The wise know their weakness too well to assume infallibility; and he who knows most, knows best how little he knows.\grqq\bigskip}%\textsc
{{Thomas Jefferson}\\ (1743--1826)}

\noindent
Nachfolgend sind einige Beispiele zum Styling von Inhalten aufgeführt.
Eine gute Einführung in das Arbeiten mit \LaTeX\ bietet die Ausarbeitung von Jürgens und Feuerstack der FernUniversität in Hagen: \url{https://www.fernuni-hagen.de/imperia/md/content/zmi_2010/a026_latex_einf.pdf}.

\indent

\section{Text}

Dies ist ein Beispiel für \textit{kursiven} und \textbf{fetten} Text.

\bigskip

Abkürzungen werden in der Datei \texttt{acronyms.tex} definiert und können dann vereinfacht genutzt werden. Alle tatsächlich eingesetzten Abkürzungen werden automatisch im Abkürzungsverzeichnis aufgeführt. Eine Abkürzung wird bei der ersten Verwendung zusätzlich ausgeschrieben dargestellt. Ein Beispiel: Das \ac{BSI} stellte fest \dots

\section{Abbildungen und Tabellen}

Eine einfache Abbildung (\ref{fig:hsd_logo}):

\begin{figure}[hbtp]
  \centering
  \includegraphics[width=.6\textwidth]{figures/hsd_m_logo.pdf}
  \caption{Logo Hochschule Düsseldorf}
  \label{fig:hsd_logo}
\end{figure}

\bigskip

Eine einfache Tabelle (\ref{tab:beispieltabelle}):

\begin{table}[hbtp]
  \begin{center}
    \begin{tabular}{|c|c|}
      \hline 
      \rule[-1ex]{0pt}{2.5ex} Eins & 1 \\ 
      \hline 
      \rule[-1ex]{0pt}{2.5ex} Zwei & 2 \\ 
      \hline
    \end{tabular}
  \end{center}
  \caption{Beispieltabelle}
  \label{tab:beispieltabelle}
\end{table}

\section{Zitieren}

Die benötigte Literatur wird in der Datei \texttt{literatur.bib} gepflegt. Das Literaturverzeichnis wird automatisch generiert.

Dies ist ein Zitat von \parencite{iso27001:2013} \dots laut \parencite[65\psqq]{iso27001:2013} ist dieses Vorgehen empfehlenswert.

\section{Listen}

Unsortierte Liste:

\begin{itemize}
 \item Eins
 \item Zwei
 \item Drei
\end{itemize}

\bigskip

\noindent
Nummerierte Liste:

\begin{enumerate}
 \item Element
 \item Element
 \item Element
\end{enumerate}

\indent