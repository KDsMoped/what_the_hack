% Kapitel
\chapter{Einleitung}

\epigraph{\glqq The user's going to pick dancing pigs over security every time.\grqq\bigskip}
{\textsc{Bruce Schneier}\\ ($\ast$1963)}

\noindent
Das Verfassen einer eigenständigen Bachelor- bzw. Masterarbeit \dots
\indent

\section{Hinweise}

Bitte lassen Sie ein Exemplar des Anmeldeformulars, welches Sie bei der Anmeldung Ihrer Arbeit im Prüfungsamt haben unterschreiben lassen und auf dem das Abgabedatum vermerkt ist, als zweite Seite dieses Dokumentes einbinden.

Bei Abgabe zeigen Sie Ihre drei Exemplare im Studienbüro vor und lassen die Abgabe auf den entsprechenden Formularen eintragen. Zwei Exemplare müssen nun persönlich den beiden Prüfern übergeben werden. Das dritte Exemplar ist für Sie bestimmt.
Bitte denken Sie auch daran, dass Sie die eidesstattliche Erklärung vor Abgabe unterschreiben.

\section{Struktur}

Die Arbeit sollte u.\,a. nachfolgende Inhalte berücksichtigen:

\begin{itemize}
  \item Titelseite
  \item Eidesstattliche Erklärung
  \item Zusammenfassung und Abstract (Englisch)
  \item Inhaltsverzeichnis, Abbildungsverzeichnis, Tabellenverzeichnis, Abkürzungsverzeichnis und Literaturverzeichnis
\end{itemize}

\noindent
Der Zitierstil sollte nach APA (American Psychological Asscociation) Style (\url{http://www.apastyle.org/}) erfolgen.

\section{Organisatorisches}

\begin{itemize}
 \item Es gilt die jeweils aktuelle Pr\"ufungsordnung (\S 15 in BMI PO vom 04.08.2010 bzw. \S 15 in MMI PO vom 16.06.2011).
 \item Abzugeben gebunden als Ausdruck und elektronisch als PDF
\end{itemize}

\section{Bewertungskriterien}

Die Bewertung einer Arbeit erfolgt unter anderem auf Grundlage von \textbf{Schwierigkeitsgrad}, \textbf{wissenschaftlicher Arbeitstechnik}, \textbf{ingenieurmäßiger Vorgehensweise}, \textbf{Stil} und \textbf{Form}.

\noindent
Das zugehörige Kolloquium wird vor allem basierend auf der \textbf{Wiedergabe der Inhalte}, der \textbf{Foliengestaltung}, \textbf{Stil} und \textbf{Form} bewertet.