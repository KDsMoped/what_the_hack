\documentclass[oneside,12pt,a4paper]{book}

% [ Packages ]
\usepackage{geometry}
\usepackage{graphicx}
\usepackage{hyperref}

% UTF8 für Umlaute
\usepackage[utf8]{inputenc}

% Formale Zitate
\usepackage{csquotes}

% Neue deutsche Rechtschreibung
\usepackage[ngerman]{babel}

% Zitate unter Überschriften darstellen
\usepackage{epigraph}

% Zeilenabstand
\usepackage{setspace}

% Akronyme
\usepackage[
  printonlyused
]{acronym}

% Kapitel / Überschriften und Seitenzahlen im Kopf anzeigen
\usepackage{fancyhdr}

% BibLaTeX mit Biber
\usepackage[
  style=apa
  ,backend=biber
  ,apabackref=true
  ,language=auto
  ,abbreviate=true
  ,maxcitenames=2
  ,natbib=true
  ,hyperref=true
  ,maxbibnames=99
  ,url=true
]{biblatex}

% Sprache für APA zuweisen
\DeclareLanguageMapping{ngerman}{ngerman-apa}

% [ Bibliography ]
\bibliography{bib/literatur}

% [ Konfiguration ]
\include{misc/config}

% [ Benutzerangaben ]
% [ Angaben zur Person ]
% Vorname
\newcommand{\DocumentAuthorPrename}{[Max]}
% Nachname
\newcommand{\DocumentAuthorName}{[Mustermann]}
% Straße und Hausnummer
\newcommand{\DocumentAuthorStreet}{[Musterstraße 67]}
% PLZ
\newcommand{\DocumentAuthorZip}{[123456]}
% Ort
\newcommand{\DocumentAuthorCity}{[Musterstadt]}
% E-Mail-Adresse
\newcommand{\DocumentAuthorEmail}{[vorname.nachname\(@\)hs-duesseldorf.de]}
% Studiengang
\newcommand{\DocumentThesisCourse}{[Medieninformatik/Medientechnik]}
% Matrikel-Nr.
\newcommand{\DocumentThesisMatr}{[123456]}

% [ Angaben zur Arbeit ]
% Art der Abschlussarbeit
\newcommand{\DocumentThesisType}{Wissenschaftliche Vertiefung und Bachelorarbeit}
% Titel der Abschlussarbeit
\newcommand{\DocumentThesisTitle}{[Titel der Wiss. Vertiefung und Bachelorarbeit]}
% Untertitel der Abschlussarbeit
\newcommand{\DocumentThesisSubtitle}{[Untertitel der Wiss. Vertiefung und Bachelorarbeit]}
% Zu erlangender akademischer Grad
\newcommand{\DocumentThesisDegree}{[Bachelor of Science]}
% Datum auf Cover
\newcommand{\DocumentThesisDateCover}{[Januar 1970]}
% Erstprüfer
\newcommand{\DocumentThesisFirstExaminer}{Prof. Dr.-Ing. Holger Schmidt}
% Zweitprüfer
\newcommand{\DocumentThesisSecondExaminer}{[Prof. Dr. Maxi Musterfrau]}

\begin{document}

  % Cover
  \include{misc/cover_wv_und_bsc}

  % Seitenzahlen röm.
  \pagenumbering{roman}

  % Eidesstattliche Erklärung
  \include{misc/eidesstattliche_erklaerung}

  % Abstract
  \include{content/abstract}

  % Inhaltsverzeichnis

  \pagestyle{plain}
  \tableofcontents

  % Abbildungsverzeichnis
  \listoffigures

  % Tabellenverzeichnis
  \listoftables

  % Abkürzungsverzeichnis
  \newpage
  \input{misc/acronyms}

  \newpage

  % Seitenzahlen arab.
  \pagenumbering{arabic}
  \pagestyle{fancy}
  
  \part{Wissenschaftliche Vertiefung}
  
  \part{Bachelorarbeit}

  % Kapitel 1 -- Einleitung
  % Kapitel
\chapter{Einleitung}

\epigraph{\glqq The user's going to pick dancing pigs over security every time.\grqq\bigskip}
{\textsc{Bruce Schneier}\\ ($\ast$1963)}

\noindent
Das Verfassen einer eigenständigen Bachelor- bzw. Masterarbeit \dots
\indent

\section{Hinweise}

Bitte lassen Sie ein Exemplar des Anmeldeformulars, welches Sie bei der Anmeldung Ihrer Arbeit im Prüfungsamt haben unterschreiben lassen und auf dem das Abgabedatum vermerkt ist, als zweite Seite dieses Dokumentes einbinden.

Bei Abgabe zeigen Sie Ihre drei Exemplare im Studienbüro vor und lassen die Abgabe auf den entsprechenden Formularen eintragen. Zwei Exemplare müssen nun persönlich den beiden Prüfern übergeben werden. Das dritte Exemplar ist für Sie bestimmt.
Bitte denken Sie auch daran, dass Sie die eidesstattliche Erklärung vor Abgabe unterschreiben.

\section{Struktur}

Die Arbeit sollte u.\,a. nachfolgende Inhalte berücksichtigen:

\begin{itemize}
  \item Titelseite
  \item Eidesstattliche Erklärung
  \item Zusammenfassung und Abstract (Englisch)
  \item Inhaltsverzeichnis, Abbildungsverzeichnis, Tabellenverzeichnis, Abkürzungsverzeichnis und Literaturverzeichnis
\end{itemize}

\noindent
Der Zitierstil sollte nach APA (American Psychological Asscociation) Style (\url{http://www.apastyle.org/}) erfolgen.

\section{Organisatorisches}

\begin{itemize}
 \item Es gilt die jeweils aktuelle Pr\"ufungsordnung (\S 15 in BMI PO vom 04.08.2010 bzw. \S 15 in MMI PO vom 16.06.2011).
 \item Abzugeben gebunden als Ausdruck und elektronisch als PDF
\end{itemize}

\section{Bewertungskriterien}

Die Bewertung einer Arbeit erfolgt unter anderem auf Grundlage von \textbf{Schwierigkeitsgrad}, \textbf{wissenschaftlicher Arbeitstechnik}, \textbf{ingenieurmäßiger Vorgehensweise}, \textbf{Stil} und \textbf{Form}.

\noindent
Das zugehörige Kolloquium wird vor allem basierend auf der \textbf{Wiedergabe der Inhalte}, der \textbf{Foliengestaltung}, \textbf{Stil} und \textbf{Form} bewertet.

  % Kapitel 2 -- Stile
  % Kapitel
\chapter{Stile}

\epigraph{\glqq The wise know their weakness too well to assume infallibility; and he who knows most, knows best how little he knows.\grqq\bigskip}%\textsc
{{Thomas Jefferson}\\ (1743--1826)}

\noindent
Nachfolgend sind einige Beispiele zum Styling von Inhalten aufgeführt.
Eine gute Einführung in das Arbeiten mit \LaTeX\ bietet die Ausarbeitung von Jürgens und Feuerstack der FernUniversität in Hagen: \url{https://www.fernuni-hagen.de/imperia/md/content/zmi_2010/a026_latex_einf.pdf}.

\indent

\section{Text}

Dies ist ein Beispiel für \textit{kursiven} und \textbf{fetten} Text.

\bigskip

Abkürzungen werden in der Datei \texttt{acronyms.tex} definiert und können dann vereinfacht genutzt werden. Alle tatsächlich eingesetzten Abkürzungen werden automatisch im Abkürzungsverzeichnis aufgeführt. Eine Abkürzung wird bei der ersten Verwendung zusätzlich ausgeschrieben dargestellt. Ein Beispiel: Das \ac{BSI} stellte fest \dots

\section{Abbildungen und Tabellen}

Eine einfache Abbildung (\ref{fig:hsd_logo}):

\begin{figure}[hbtp]
  \centering
  \includegraphics[width=.6\textwidth]{figures/hsd_m_logo.pdf}
  \caption{Logo Hochschule Düsseldorf}
  \label{fig:hsd_logo}
\end{figure}

\bigskip

Eine einfache Tabelle (\ref{tab:beispieltabelle}):

\begin{table}[hbtp]
  \begin{center}
    \begin{tabular}{|c|c|}
      \hline 
      \rule[-1ex]{0pt}{2.5ex} Eins & 1 \\ 
      \hline 
      \rule[-1ex]{0pt}{2.5ex} Zwei & 2 \\ 
      \hline
    \end{tabular}
  \end{center}
  \caption{Beispieltabelle}
  \label{tab:beispieltabelle}
\end{table}

\section{Zitieren}

Die benötigte Literatur wird in der Datei \texttt{literatur.bib} gepflegt. Das Literaturverzeichnis wird automatisch generiert.

Dies ist ein Zitat von \parencite{iso27001:2013} \dots laut \parencite[65\psqq]{iso27001:2013} ist dieses Vorgehen empfehlenswert.

\section{Listen}

Unsortierte Liste:

\begin{itemize}
 \item Eins
 \item Zwei
 \item Drei
\end{itemize}

\bigskip

\noindent
Nummerierte Liste:

\begin{enumerate}
 \item Element
 \item Element
 \item Element
\end{enumerate}

\indent

  % Kapitel 3 -- Tooling
  % Kapitel
\chapter{Tools}

\epigraph{\glqq Man is still the most extraordinary computer of all.\grqq\bigskip}%\textsc
{{John F. Kennedy}\\ (1917--1963)}

\noindent
Nachfolgende Hinweise und Empfehlungen zum Einsatz von Tools vereinfachen den Umgang mit \LaTeX\ und Git.
\indent

\section{\LaTeX}

Das Verfassen von Dokumenten mit \LaTeX\ kann durch unterschiedlichste Tools unterstützt werden. Da \LaTeX\ grundsätzlich textbasiert arbeitet können jegliche Inhalte auch in einem einfachen Texteditor erstellt und angepasst werden.\\

Mittels unterschiedlichster Editoren kann die Erstellung und Pflege von Dokumenten mit \LaTeX\ vereinfacht werden. Unter Linux bietet der Editor „Kile“ (\url{http://kile.sourceforge.net/}) eine Vielzahl nützlicher Funktionen. Für Apple OS X und Microsoft Windows ist „Texmaker“ (\url{http://www.xm1math.net/texmaker/}) empfehlenswert.
  
    % Kapitel 4 -- Infrastruktur
  \include{content/kapitel_4}

  % Anhang A
  \include{content/anhang_a}

  % Literaturverzeichnis
  \printbibliography

\end{document}